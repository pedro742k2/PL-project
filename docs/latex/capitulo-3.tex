\chapter{Tecnologias e Ferramentas Utilizadas}
% OU \chapter{Trabalhos Relacionados}
% OU \chapter{Engenharia de Software}
% OU \chapter{Tecnologias e Ferramentas Utilizadas}
\label{chap:tecno-ferra}

\section{Introdução}
\label{chap3:sec:intro}
No decorrer deste projeto, foram usadas varias ferramentas para a pesquisa, escrita e desenvolvimento do mesmo, vitais para o bom-funcionamento e correta execução dos programas desenvolvidos. Estes recursos irão ser especificados de acordo com a Tabela 3.1 com mais detalhe nas secções posteriores.

\begin{table}
\centering
\begin{tabular}{|c|c|}
\hline
\textbf{Recurso} & \textbf{Sub-Secção}\\
\hline
\hline
Linux & 3.2.1 \\
\hline	
VSC & 3.2.2 \\
\hline
Overleaf & 3.2.3 \\
\hline
\end{tabular}
\caption{Ferramentas Utilizadas}
\label{tab:tools}
\end{table}

\section{Recursos}
\label{chap3:rec:int}

\subsection{Linux}
\label{chap3:rec:linux}
O projeto na sua totalidade foi desenvolvido em ambiente Linux, usando a distribuição Ubuntu 21.10.
A sua execução está optimizada para distribuições semelhantes visto que todos os testes foram realizados neste ambiente. Dentro do Sistema Operativo, através do terminal, foi utilizado diverso software para o bom funcionamento do programa.

\subsection{VSC}
\label{chap3:rec:vsc}
Enquanto o projeto foi desenvolvido em ambiente Linux, neste foi utilizado o Visual Studio Code para a escrita e para garantir uma boa prática de código C. Este compilador, ao ser bastante interactivo, permite uma organização superior do código e debugging mais eficiente.

\subsection{Overleaf}
\label{chap3:rec:overleaf}
Após a conclusão do projeto, este mesmo relatório final sobre o desenvolvimento do projeto foi desenvolvido com o uso do Software \LaTeX. \\
O uso do mesmo, permite um relatório eficiente e formal, alcançando o objetivo pretendido.

\section{Conclusões}
\label{chap3:sec:concs}
Através do uso dos diferentes recursos, foi possível originar um projeto completo e nutrido. \\
Ao utilizar estas ferramentas, não só acaba por ficar um trabalho melhor desenvolvido, mas também aumenta a habilidade dos envolvidos a trabalhar nos ambientes mencionados. 